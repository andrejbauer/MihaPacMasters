% !TEX TS-program = pdflatex
% !TEX encoding = UTF-8 Unicode

% This is a simple template for a LaTeX document using the "article" class.
% See "book", "report", "letter" for other types of document.

\documentclass[11pt]{article} % use larger type; default would be 10pt

\usepackage[utf8]{inputenc} % set input encoding (not needed with XeLaTeX)
\usepackage{hyperref}

%%% Examples of Article customizations
% These packages are optional, depending whether you want the features they provide.
% See the LaTeX Companion or other references for full information.

%%% PAGE DIMENSIONS
\usepackage{geometry} % to change the page dimensions
\geometry{a4paper} % or letterpaper (US) or a5paper or....
% \geometry{margin=2in} % for example, change the margins to 2 inches all round
% \geometry{landscape} % set up the page for landscape
%   read geometry.pdf for detailed page layout information

\usepackage{graphicx} % support the \includegraphics command and options

% \usepackage[parfill]{parskip} % Activate to begin paragraphs with an empty line rather than an indent

%%% PACKAGES
\usepackage{booktabs} % for much better looking tables
\usepackage{array} % for better arrays (eg matrices) in maths
\usepackage{paralist} % very flexible & customisable lists (eg. enumerate/itemize, etc.)
\usepackage{verbatim} % adds environment for commenting out blocks of text & for better verbatim
\usepackage{subfig} % make it possible to include more than one captioned figure/table in a single float
% These packages are all incorporated in the memoir class to one degree or another...

%%% HEADERS & FOOTERS
\usepackage{fancyhdr} % This should be set AFTER setting up the page geometry
\pagestyle{fancy} % options: empty , plain , fancy
\renewcommand{\headrulewidth}{0pt} % customise the layout...
\lhead{}\chead{}\rhead{}
\lfoot{}\cfoot{\thepage}\rfoot{}

%%% SECTION TITLE APPEARANCE
\usepackage{sectsty}
\allsectionsfont{\sffamily\mdseries\upshape} % (See the fntguide.pdf for font help)
% (This matches ConTeXt defaults)

%%% ToC (table of contents) APPEARANCE
\usepackage[nottoc,notlof,notlot]{tocbibind} % Put the bibliography in the ToC
\usepackage[titles,subfigure]{tocloft} % Alter the style of the Table of Contents
\renewcommand{\cftsecfont}{\rmfamily\mdseries\upshape}
\renewcommand{\cftsecpagefont}{\rmfamily\mdseries\upshape} % No bold!

%%% END Article customizations

%%% The "real" document content comes below...

\title{Masters Thesis}
\author{Mihael Pačnik}
%\date{} % Activate to display a given date or no date (if empty),
         % otherwise the current date is printed 

\begin{document}
\maketitle

\section{Introduction}
An explanation of what the project was, which was writing a formalization of the co-op language, what the co-op language is, why this is important and what was accomplished.
%A short story of what this even is, and why it would be important, and what was accomplished.

%Leta 2022 sem potreboval izbrati magistrsko temo, in, ker je vedno bolj aktualno računalništvo v našem svetu, z vsakdanjimi napredki in rešitvami, ki se dosežejo prav preko računalniški in programerskih postopkov, sem se odločil, da bom delal na temi, ki se ukvarja ravno s tem. Z napredkom v vseh metodah, naj bodo te industrijske ali zgolj teoretične, težijo te metode k enostavnosti. 
%
%Računalniški jeziki, dandanes, so ustvarjeni tako, da ima programer veliko svobode, kaj ali kako hoče uporabiti resurze računalnika. Vendar ta svoboda pogosto prevede do napak, naj bodo te da program zahteva pomnilnik, ko le-tega ni na voljo, ali hoče shraniti podatke na trdi disk, ko je ta poln. Ali tudi bolj zapleteni problemi, ko se pojavijo takrat, ko program hoče odpreti datoteko, ki jo drug program uporablja. Co-op jezik, ki bo razložen v teh straneh, je izdelan, da se programerju ali programerki ni treba obadati s takimi detajli, vendar pa še vedno dopušča dovolj svobode, da se lahko ustvari, karkoli se že potrebuje.
%
%To delo je formalizacija dela mojega mentorja Andreja Bauerja in somentorja Danela Ahmana, "Runners in action", kjer predstavita na teoretičen način programski jezik, ki reši ta problem. Moje delo je bilo implementirati ta jezik v programskem jeziku Agda, in ga v tem jeziku tudi dokazati v eni interpretaciji.

%LOOK AT THE ARXIV PAPER FOR INSPIRATION

% SET UP the semantics first?

%Equational theory? 

%Language and code together or first the language and then code? Harder to do first, but if you separate the two you'll do parts twice, and 
%You can do it inline. Best to do it together? Or at least some of the code together with the explanation. 

%Explain what are deficits of non-runner languages what can go wrong, what a runner language can do, when you explain the language in detail you intertwine 

%The agda code has a simplified version of the language:
%1. At the very beginning say that you are explaining a simplified language 
%2. Present the full version and then formalize the simplified (this is harder)
%Better to do 1. - present the simplified language 

%Certain kind of language with certain features and as you would be explaining it you show the code, 


%Introduction, non-technical background motivation 
%But then, after that, surely you would follow the logical dependencies. So you would presumably be describing telling the story so that it makes sense to humans & respects the dependency of the code. So that can be a guide for how to structure it.
%You would more or less follow in parallel your code and your runners paper, and synthesize it into a human explanation of the language. 
%Danel's thesis is a very good example of that. 


%Look at both the paper, and code. Introduction and motivation, it's okay to get your thesis from multiple sources - paper, code, general text book, programming languages, synthesize it, say it with your own words. 
%Then go and write down what would be the main parts of your agda code, and how they would they be chopped up into chapters? Because then you have the motivation, if someone asks you what did you do in your thesis, you should be able to give an answer. And this should be stated somewhere in the beginning. Formalization, obviously, but of what? That needs to get explained. 
%Thinking of it as taking the paper & agda code, and mesh them together so that the code and the text tell one story about what this language is. You could also use the agda formalization as a little implementation of this language, and try to write some examples from the paper and see what they evaluate to. You can explain that you wrote down semantics, but it is also an interpreter from co-op to agda, and you can demonstrate how things work with some of the examples. Or use examples from your imagination to tell what you want to illustrate or explain, or just to explore the formalization. Can you actually compute the interpretation of this in agda and normalize, do you actually get what you think you should be getting and so on?
%Examples are a good opportunity to contribute to the thesis.
%Program, of type int, compute its semantics, and normalize that, you must get a number out of it. And that is an experiment you can perform and show that it works.
%A different example is not a complete program, but maybe something that computes to a particular piece of monadic code, and we're passing state how we thought it would, or this is how you pass state, etc.

%If you look at the paper, the examples there, are just written on pen and paper - there is the ocaml reference, but now if you express them in agda then the fact that you can write the example in agda, it means that it is actually well-typed.



%If you order your agda code so that you have a linear order of dependencies, then you explain what your code is, then you have your order of chapters and thesis. Explain the language + math in the code rather, not the literal code. Match the order of dependencies against the paper. Presumably substitutions and renamings aren't going to be particularly helpful for this since they are standard.

%Two chapters:
%1. Language & Equations
%2. Model/Interpreter
%3. Examples. Now that we are done let's do something with this.
%Could chop it up more, if you wanted to. Or you could literally just have four chapters. One place where to chop it up would be where the interpretation and such with the soundness proofs. Depends how in-depth you want to go with proving, how you explain them without dumping all the agda code in there. 


%When cleaning up the code make sure that you don't import more that you don't need. Craft.sh that can build a tree of dependencies. 
%Validity = soundness (change name), denotation is interpreter and so on - better names for these things.
%Keep this first outline around as you never know if the ideas might come in handy.

\section{New order in homage to the famous 1980's band}
%these will probably get subsections, and know which agda file is which section
%Go through the language, denotational semantics, soundness, examples section and ascribe agda files to them. Write down the subsection names, should be clear what files that will be about, and as you do this, you will see what imports you can clean up, and rename the files into more descriptive ones (sub and ren-validity are terrible names)
%Create chapters out of the items not sections
%Write also "Here we are going to explain..." sort of stuff while you're writing your outline. 
\begin{enumerate}
	\item Introduction \& Motivation (Define a mapping from language to agda, by the way this is going to be used in other chapter to actually compute with this)
	%motivation of why the coop language was conceived, and then my motivation. Why is it even worthwhile to formalize it, and what the language is? What the language is trying to solve and then come to your motivations and contributions.	
	\item Background (on Math and or Agda) %Read the paper and see how it was motivated, and think about what your motivation will be - since yours will be formalization
	%Once you start thinking about how to explain the later chapters, you will find backgrounds. Do this as you go through the agda code (probably). Make a plan when you go through the agda code, BEFORE writing.
	\item The Language: Types, terms, substitution, syntax, equational rules 
	\item Denotational Semantics (emphasize this is also an interpreter)
	\item Soundness of equational rules
	\item Examples (or some sort of evaluation of what happened/case study) (Looking at what you've done and saying something intelligent about it)
	\item Conclusion (even if short)
%Match the agda files with these, which will also show the order
\end{enumerate}


%Write also short sentences for each part while you're writing

\section{Co-op language}
A longer explanation of what the co-op language is, why it was created, how it is structured. Perhaps how in general languages are defined or what they need to be?
%An introduction into what the co-op language is and why it exists.

\subsection{What are (co)operations?}

\subsection{What are runners?}%Kaj so runner-ji?}


\section{Denotacijska semantika}
What are denotational semantics and what are they for this project specifically.
%Explaining what it is in general would be a paragraph or a whole book

%Kaj to je, kaj pomeni za projekt.


\section{Formalization}%Formalizacija}
What formalization of a language is, how it was done, why it is important.
%The reason behind, the use of, and the method by which the co-op language as written by my mentor and co-mentor was formalized in agda code, and what a formalization entails.


\section{Agda}
An introduction to the programming language of agda and its special characteristics, why was it chosen for this project, what it means to prove something in agda and the other things that were done in the code such as equivalence.
%Agda je funkcijski programski jezik? podoben v strukturi Haskell-u ipd. %Več o Agdi bo treba vedeti za to
%Kaj pomeni "dokazati" nekaj v agdi, kako je strukturirana koda, kaj zares pomeni, da sta dve stvari "ekvivalentni" in drugi pojmi v kodi.

\section{Structure of Code}%Struktura Kode}
A section that will explain the code of the project in detail, going through the separate .agda files and explaining them.
%Projekt, ki je razdeljen v različne .agda datoteke zaradi preglednosti in hitrosti (kajti daljša kot je datoteka, dlje traja, da se naloži), je skonstruiran v tri faze. 
%Prva definira ta osnovne tipe, ki se uporabljajo vseskozi, ti so
Three distinct phases of code, the first of which defines the basic types and such.
\begin{itemize}
	\item terms 
	\item types
	\item contexts
	\item parameters
\end{itemize}

%Naslednja faza se tiče formalizacije co-op jezika v agdo, te bodo 
Next phase which is more about the formalization of the co-op language,
\begin{itemize}
	\item equations
	\item renaming
	\item substition
\end{itemize}
Finally the files which define the interpretation of the co-op language.
%Nazadnje so datoteke, ki definirajo eno interpretacijo co-op jezika, in s tem dokažejo veljavnost vse ostale kode 
\begin{itemize}
	\item Sub-validity
	\item ren-validity
	\item validity
	\item monads
	\item denotations
\end{itemize}


\section{Zaključek}
More in-depth explanation of what was achieved, what this all served to, mentioning of how it fixed a small error in the original formulation. Potential further work?
%Razlaga, kaj je uspelo ustvariti, čemu to služi, omemba popravka napake zaradi formalizacije. Potencialno nadaljne delo?


\section{Bibliography}
Important facet also to have all of your sources in here, somewhere, so you can read them, and rely on them. 

\url{https://repozitorij.uni-lj.si/Dokument.php?id=174686&lang=eng} <- za inspiracijo glede strukture (in formalizacije)
\url{https://arxiv.org/pdf/1910.11629} <- Runners in action, D. Ahman, A. Bauer
\url{https://www.eff-lang.org/handlers-tutorial.pdf} <- An introduction to algebraic effects, M. Pretnar
\url{https://cs.ioc.ee/~tarmo/mgs21/mgs1.pdf} <- Monads \& Interaction
\url{https://danel.ahman.ee/papers/mfps13.pdf} <- Danel Ahman Master's Thesis

2. Sam Lindley’s Lecture 1 from last year’s Oregon Programming Language summer school. The video and the slides are available at \url{https://www.cs.uoregon.edu/research/summerschool/summer22/topics.php#Lindley}:

   * video: \url{https://www.cs.uoregon.edu/research/summerschool/summer22/lectures/Lindley1.mp4}
   * slides: \url{https://www.cs.uoregon.edu/research/summerschool/summer22/lectures/handlers.pdf}
\url{https://danel.ahman.ee/papers/acs-dissertation.pdf} Actual Danel's Master's Thesis

\end{document}
